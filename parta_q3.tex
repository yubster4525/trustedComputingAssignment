\section{Question 3: TEE Components}

\subsection*{Question}
\textit{Explain the core components and operational flow of a Trusted Execution Environment (TEE), using ARM TrustZone or Intel SGX as examples.}

\subsection{Introduction}

A Trusted Execution Environment (TEE) is an isolated execution environment that provides security features such as isolated execution, integrity of applications, and confidentiality of data. TEE enables secure execution alongside a rich OS while providing strong isolation guarantees.

\subsection{ARM TrustZone Architecture}

\subsubsection{Core Components}

\paragraph{Processor Security Extensions}

\textbf{Secure Monitor Mode (EL3)}
\begin{itemize}[leftmargin=*]
    \item Highest privilege level in ARMv8-A architecture
    \item Acts as gatekeeper between Normal and Secure Worlds
    \item Handles world context switches
    \item Routes exceptions and interrupts to appropriate world
\end{itemize}

\textbf{Security State Bit (NS bit)}
\begin{itemize}[leftmargin=*]
    \item NS=0: Secure World
    \item NS=1: Normal World
    \item All processor states tagged with NS bit
\end{itemize}

\paragraph{Memory Protection}

\textbf{TrustZone Address Space Controller (TZASC)}
\begin{itemize}[leftmargin=*]
    \item Filters memory transactions
    \item Marks regions as Secure or Non-Secure
    \item Normal World access to Secure memory $\rightarrow$ Blocked (bus error)
    \item Secure World can access both regions
\end{itemize}

\textbf{TrustZone Protection Controller (TZPC)}
\begin{itemize}[leftmargin=*]
    \item Controls peripheral access
    \item Tags peripherals as Secure or Non-Secure
    \item Example: Crypto engine = SECURE, UART = NON-SECURE
\end{itemize}

\paragraph{Software Stack}

\begin{itemize}[leftmargin=*]
    \item \textbf{Normal World}: Rich OS (Linux, Android), untrusted applications
    \item \textbf{Secure World}: Trusted OS (OP-TEE, Trusty), Trusted Applications
    \item \textbf{Secure Monitor}: ARM Trusted Firmware (ATF)
\end{itemize}

\subsubsection{Operational Flow: Secure Mobile Payment}

\textbf{Scenario}: Mobile banking app needs to sign payment using key never exposed to Normal World.

\begin{enumerate}[leftmargin=*]
    \item User clicks "Pay \$100" in Normal World banking app
    \item App prepares transaction data
    \item \texttt{TEEC\_InvokeCommand()} triggers SMC instruction
    \item \textbf{World Switch}: Secure Monitor saves Normal World context
    \item Switch to Secure World, restore Secure World context
    \item OP-TEE Kernel dispatches to Payment Trusted Application (TA)
    \item TA:
    \begin{itemize}
        \item Validates transaction
        \item Loads signing key from secure storage
        \item Signs transaction with secure crypto engine
        \item Returns signed result
    \end{itemize}
    \item \textbf{World Switch Back}: Secure Monitor context switch
    \item Normal World receives signed transaction
    \item App sends to bank server
\end{enumerate}

\textbf{Security Guarantee}: Private key never exposed to Normal World, even if Android OS compromised.

\subsection{Intel SGX Architecture}

\subsubsection{Core Components}

\paragraph{Processor Reserved Memory (PRM)}

\textbf{Encrypted Page Cache (EPC)}
\begin{itemize}[leftmargin=*]
    \item Special DRAM region (typically 128-256MB)
    \item Stores enclave code and data
    \item Managed exclusively by CPU
    \item Encrypted in DRAM using Memory Encryption Engine (MEE)
    \item Plaintext exists only inside CPU package
\end{itemize}

\textbf{EPC Map (EPCM)}
\begin{itemize}[leftmargin=*]
    \item Metadata table tracking EPC pages
    \item One entry per 4KB page
    \item Fields: valid, read, write, execute, page type, enclave owner
    \item Enforced by hardware on every memory access
\end{itemize}

\paragraph{Memory Encryption Engine (MEE)}

\begin{itemize}[leftmargin=*]
    \item Encrypts/decrypts EPC pages in real-time
    \item AES-128 in counter mode
    \item 56-bit MAC for integrity
    \item Merkle tree for replay protection
    \item Plaintext only in CPU cache
    \item DRAM contains only ciphertext
\end{itemize}

\paragraph{SGX Instructions}

\textbf{Enclave Management (Ring 0)}:
\begin{itemize}[leftmargin=*]
    \item \texttt{ECREATE}: Create new enclave
    \item \texttt{EADD}: Add page to enclave
    \item \texttt{EEXTEND}: Measure enclave page (builds MRENCLAVE)
    \item \texttt{EINIT}: Initialize enclave
\end{itemize}

\textbf{Enclave Execution (Ring 3)}:
\begin{itemize}[leftmargin=*]
    \item \texttt{EENTER}: Enter enclave
    \item \texttt{EEXIT}: Exit enclave
    \item \texttt{ERESUME}: Resume after interrupt
    \item \texttt{EREPORT}: Generate enclave report
\end{itemize}

\paragraph{Enclave Measurement (MRENCLAVE)}

\begin{lstlisting}[caption=MRENCLAVE Calculation]
Initial_Hash = SHA256("")

For each page added (EADD):
    For each 256-byte chunk (EEXTEND):
        Current_Hash = SHA256(
            Current_Hash ||
            chunk_data ||
            page_offset ||
            security_info
        )

Final_MRENCLAVE = Current_Hash
\end{lstlisting}

\textbf{Properties}:
\begin{itemize}[leftmargin=*]
    \item 256-bit SHA-256 hash
    \item Uniquely identifies enclave contents and layout
    \item Used for attestation and sealing keys
\end{itemize}

\subsubsection{Operational Flow: Confidential ML Inference}

\textbf{Scenario}: Cloud provides ML inference. Client has sensitive medical images.

\textbf{Setup Phase}:
\begin{enumerate}[leftmargin=*]
    \item ML provider develops enclave code
    \item Builds and measures: MRENCLAVE = a3f8c2...b4d6
    \item Publishes MRENCLAVE (clients will verify this)
\end{enumerate}

\textbf{Runtime Flow}:
\begin{enumerate}[leftmargin=*]
    \item Server creates SGX enclave (\texttt{ECREATE, EADD, EEXTEND, EINIT})
    \item Client requests attestation
    \item Enclave generates report with MRENCLAVE
    \item Quoting Enclave signs report $\rightarrow$ Quote
    \item Client verifies:
    \begin{itemize}
        \item Intel signature on quote
        \item MRENCLAVE matches expected value
    \end{itemize}
    \item Client establishes secure channel (TLS to enclave)
    \item Client encrypts medical image with session key
    \item Server: \texttt{EENTER} $\rightarrow$ Enclave decrypts image inside CPU
    \item ML inference performed in enclave
    \item Result encrypted, \texttt{EEXIT} returns to untrusted code
    \item Client receives and decrypts result
\end{enumerate}

\textbf{Security Guarantee}: Data encrypted in DRAM, protected from malicious OS/hypervisor, physical memory attacks.

\subsection{Comparison: TrustZone vs SGX}

\begin{table}[h]
\centering
\begin{tabular}{|p{3cm}|p{5.5cm}|p{5.5cm}|}
\hline
\textbf{Aspect} & \textbf{ARM TrustZone} & \textbf{Intel SGX} \\
\hline
Isolation & System-wide (two worlds) & Per-application (enclaves) \\
\hline
TCB Size & Larger (Trusted OS) & Smaller (enclave only) \\
\hline
Memory & TZASC/TZPC (MB-GB) & MEE encryption (limited EPC) \\
\hline
Attestation & Limited & Built-in remote attestation \\
\hline
Secure UI & Yes (secure display) & No (no I/O in enclave) \\
\hline
Use Cases & Mobile payments, DRM & Cloud confidential computing \\
\hline
\end{tabular}
\caption{TrustZone vs SGX Comparison}
\end{table}

\subsection{Conclusion}

ARM TrustZone and Intel SGX represent two distinct TEE approaches. TrustZone provides system-wide isolation with two parallel worlds, suitable for mobile scenarios requiring secure UI. SGX offers fine-grained, per-application isolation with strong cryptographic guarantees, ideal for cloud confidential computing. Both enable secure computation in untrusted environments with different trust models and performance characteristics.

\newpage
