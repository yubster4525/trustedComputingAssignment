\section{Question 3: TEE Components - ARM TrustZone and Intel SGX}

\subsection{ARM TrustZone Architecture}

TrustZone creates two parallel execution environments on a single CPU:

\subsubsection{Core Components}

\paragraph{Processor Security Extensions}
\begin{itemize}[leftmargin=*]
    \item \textbf{Secure Monitor (EL3)}: Highest privilege level; mediates transitions between worlds via SMC instruction
    \item \textbf{Security State Bit (NS)}: NS=0 (Secure World), NS=1 (Normal World)
    \item \textbf{Dual Worlds}:
    \begin{itemize}
        \item Normal World: Rich OS (Android, Linux), untrusted apps
        \item Secure World: Trusted OS (OP-TEE), Trusted Applications (TAs)
    \end{itemize}
\end{itemize}

\paragraph{Memory Protection}
\begin{itemize}[leftmargin=*]
    \item \textbf{TZASC (Address Space Controller)}: Marks memory as Secure/Non-Secure; Normal World access to Secure memory $\rightarrow$ Bus error
    \item \textbf{TZPC (Protection Controller)}: Tags peripherals (e.g., Crypto engine = SECURE, UART = NON-SECURE)
\end{itemize}

\subsubsection{Operational Flow: Mobile Payment}

\textbf{Scenario}: Banking app signs payment transaction without exposing key to Normal World OS.

\begin{enumerate}[leftmargin=*]
    \item User initiates payment (Normal World)
    \item App calls \texttt{TEEC\_InvokeCommand()} $\rightarrow$ SMC instruction
    \item \textbf{World Switch}: Secure Monitor saves Normal World context, switches to Secure World
    \item OP-TEE dispatches to Payment TA
    \item TA: Validates transaction, loads key from secure storage, signs with crypto engine
    \item \textbf{World Switch Back}: Return to Normal World with signed transaction
    \item App sends to bank server
\end{enumerate}

\textbf{Security}: Private key never exposed to Normal World; even if Android compromised, key remains safe.

\subsection{Intel SGX Architecture}

SGX provides per-application isolation via encrypted memory regions (enclaves).

\subsubsection{Core Components}

\paragraph{Encrypted Page Cache (EPC)}
\begin{itemize}[leftmargin=*]
    \item Special DRAM region (128-256MB) managed exclusively by CPU
    \item Encrypted by Memory Encryption Engine (MEE) using AES-128
    \item Plaintext exists only inside CPU package; DRAM contains only ciphertext
    \item Protected from malicious OS, hypervisor, physical memory attacks
\end{itemize}

\paragraph{SGX Instructions}
\begin{itemize}[leftmargin=*]
    \item \textbf{Management} (Ring 0): \texttt{ECREATE} (create enclave), \texttt{EADD} (add page), \texttt{EEXTEND} (measure page), \texttt{EINIT} (finalize)
    \item \textbf{Execution} (Ring 3): \texttt{EENTER} (enter enclave), \texttt{EEXIT} (exit enclave)
\end{itemize}

\paragraph{Enclave Measurement (MRENCLAVE)}
\begin{lstlisting}
For each page added:
    For each 256-byte chunk:
        MRENCLAVE = SHA256(MRENCLAVE || chunk || offset)
\end{lstlisting}

256-bit hash uniquely identifying enclave code/data; used for attestation and sealing.

\subsubsection{Operational Flow: Confidential ML Inference}

\textbf{Scenario}: Cloud ML service processes sensitive medical images without exposing to provider.

\begin{enumerate}[leftmargin=*]
    \item \textbf{Setup}: Provider builds enclave, publishes MRENCLAVE = a3f8c2...b4d6
    \item \textbf{Runtime}:
    \begin{itemize}
        \item Server creates enclave (\texttt{ECREATE, EADD, EINIT})
        \item Client requests attestation
        \item Enclave generates report; Quoting Enclave signs $\rightarrow$ Quote
        \item Client verifies Intel signature + MRENCLAVE match
        \item Secure channel established (TLS to enclave)
    \end{itemize}
    \item Client encrypts medical image, sends to server
    \item \texttt{EENTER}: Decrypt inside enclave, run ML inference, encrypt result, \texttt{EEXIT}
    \item Client decrypts result
\end{enumerate}

\textbf{Security}: Data encrypted in DRAM; cloud provider cannot see plaintext.

\subsection{Comparison: TrustZone vs SGX}

\begin{table}[h]
\centering
\small
\begin{tabular}{|l|p{5cm}|p{5cm}|}
\hline
\textbf{Aspect} & \textbf{ARM TrustZone} & \textbf{Intel SGX} \\
\hline
Isolation & System-wide (two worlds) & Per-application (enclaves) \\
\hline
TCB Size & Larger (Trusted OS) & Smaller (enclave only) \\
\hline
Memory & TZASC/TZPC (MB-GB) & MEE encryption (limited EPC) \\
\hline
Attestation & Limited (vendor-specific) & Built-in (Intel IAS) \\
\hline
Secure UI & Yes (secure display/touch) & No (no I/O in enclave) \\
\hline
Use Cases & Mobile payments, DRM, biometrics & Cloud confidential computing \\
\hline
\end{tabular}
\caption{TrustZone vs SGX Comparison}
\end{table}

\subsection{Conclusion}

ARM TrustZone and Intel SGX represent distinct TEE approaches. TrustZone provides system-wide isolation suitable for mobile scenarios requiring secure UI. SGX offers fine-grained isolation with strong cryptographic guarantees, ideal for cloud confidential computing. Both enable secure execution in untrusted environments with different trust models and performance characteristics.

\newpage
