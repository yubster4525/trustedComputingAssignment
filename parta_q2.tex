\section{Question 2: TPM Based Attestation}

\subsection*{Question}
\textit{Describe how remote attestation works in TPM. Provide a use-case scenario where attestation ensures platform integrity.}

\subsection{Introduction}

Remote Attestation is a security mechanism that enables a computing platform to prove its trustworthiness to a remote party. Using TPM as a hardware root of trust, attestation provides cryptographic evidence of the platform's configuration and integrity state.

\subsection{Remote Attestation Protocol}

\subsubsection{Key Components}

\paragraph{Endorsement Key (EK)}
\begin{itemize}[leftmargin=*]
    \item 2048-bit RSA or ECC P-256 key pair
    \item Generated during TPM manufacturing, embedded in chip
    \item Never leaves the TPM
    \item Unique identifier for each TPM
    \item Certified by TPM manufacturer
\end{itemize}

\paragraph{Attestation Identity Key (AIK)}
\begin{itemize}[leftmargin=*]
    \item Privacy-preserving attestation credential
    \item Generated on-demand by TPM
    \item Acts as pseudonym for EK
    \item Private key never exported
    \item Signs attestation quotes
\end{itemize}

\paragraph{Quote Structure}
\begin{lstlisting}[language=C, caption=TPM Quote Structure]
Quote = {
    Magic: TPM_GENERATED_VALUE
    QualifiedSigner: AIK_name
    ExtraData: Nonce from verifier
    ClockInfo: TPM clock, reset counter
    FirmwareVersion: TPM firmware version
    PCRSelect: Which PCRs included
    PCRDigest: Hash of selected PCR values
}
Signature = Sign(Quote, AIK_private_key)
\end{lstlisting}

\subsubsection{Step-by-Step Attestation Flow}

\paragraph{Phase 1: AIK Provisioning (One-Time Setup)}
\begin{enumerate}[leftmargin=*]
    \item Attester generates AIK: \texttt{TPM2\_Create(AIK)}
    \item Request AIK certification from Privacy CA
    \item CA verifies EK certificate chain to manufacturer
    \item CA issues AIK certificate
    \item Attester stores AIK certificate for future use
\end{enumerate}

\paragraph{Phase 2: Remote Attestation (Repeated as Needed)}
\begin{enumerate}[leftmargin=*]
    \item Platform boots with measured boot enabled
    \item Verifier sends challenge: \texttt{\{Nonce, PCR\_selection\}}
    \item Attester reads PCR values from TPM
    \item TPM generates quote: \texttt{TPM2\_Quote(AIK, Nonce, PCRs)}
    \item Attester sends: \texttt{\{Quote, Signature, AIK\_cert, PCR\_log\}}
    \item Verifier:
    \begin{itemize}
        \item Verifies AIK certificate
        \item Verifies quote signature
        \item Validates nonce (freshness check)
        \item Compares PCR values to expected "golden" values
    \end{itemize}
    \item Access decision: GRANT or DENY
\end{enumerate}

\subsection{Security Properties}

\begin{itemize}[leftmargin=*]
    \item \textbf{Authenticity}: Evidence comes from genuine TPM hardware
    \item \textbf{Integrity}: Platform state hasn't been tampered with
    \item \textbf{Freshness}: Random nonce prevents replay attacks
    \item \textbf{Privacy}: AIK protects user identity across contexts
\end{itemize}

\subsection{Use Case: Enterprise Cloud VM Attestation}

\subsubsection{Scenario}

\textbf{TechCorp} is a financial services company using Microsoft Azure for critical workloads. They must ensure VMs are running verified, unmodified software and comply with financial regulations (PCI-DSS, SOX).

\subsubsection{Problem}

Traditional security cannot detect:
\begin{itemize}[leftmargin=*]
    \item Hypervisor-level malware
    \item Modified bootloaders (bootkits)
    \item Unauthorized kernel modules
    \item Firmware-level persistence
\end{itemize}

\textbf{Risk}: Attacker with hypervisor access could modify VM boot sequence, install rootkit, and steal financial data.

\subsubsection{Solution: TPM-Based Attestation}

\paragraph{Implementation Flow}

\begin{enumerate}[leftmargin=*]
    \item \textbf{VM Provisioning}: Azure provisions VM with virtual TPM (vTPM)

    \item \textbf{Measured Boot}: During boot, Azure measures:
    \begin{itemize}
        \item UEFI Firmware $\rightarrow$ PCR 0,7
        \item GRUB Bootloader $\rightarrow$ PCR 4
        \item Linux Kernel $\rightarrow$ PCR 8,9
        \item System Services $\rightarrow$ PCR 10-15
    \end{itemize}

    \item \textbf{Application Requests Secrets}: Finance app requests database credentials from Key Vault

    \item \textbf{Attestation Challenge}: Azure Key Vault initiates attestation:
    \begin{lstlisting}[caption=Key Vault Attestation Logic (simplified)]
if secret_name in CRITICAL_SECRETS:
    nonce = generate_nonce()
    quote = request_quote(vm_identity, nonce, PCRs)
    result = attestation_service.verify(quote, nonce)

    if result.trusted:
        return get_secret(secret_name)
    else:
        log_security_alert(vm_identity)
        raise AccessDeniedException()
    \end{lstlisting}

    \item \textbf{Quote Generation}: VM's vTPM generates quote with PCR values

    \item \textbf{Verification}: Azure Attestation Service:
    \begin{itemize}
        \item Verifies AIK certificate
        \item Verifies quote signature
        \item Validates nonce
        \item Compares PCR values to attestation policy
    \end{itemize}

    \item \textbf{Access Decision}:
    \begin{itemize}
        \item \textbf{Scenario A - Compliant VM}: PCRs match policy $\rightarrow$ Token issued $\rightarrow$ Secret released
        \item \textbf{Scenario B - Compromised VM}: PCR mismatch $\rightarrow$ Access denied $\rightarrow$ Security alert
    \end{itemize}
\end{enumerate}

\subsubsection{Attack Scenario Prevented}

\begin{enumerate}[leftmargin=*]
    \item Day 1: Attacker compromises Azure account credentials
    \item Day 2: Attacker modifies VM boot disk with malicious bootloader
    \item Day 3: VM reboots, bootloader measures differently: PCR[4] changes
    \item Day 4: Finance app requests database password
    \begin{itemize}
        \item Triggers attestation
        \item PCR[4] mismatch detected
        \item \textbf{ACCESS DENIED}
        \item Security Operations Center alerted
        \item VM automatically isolated
    \end{itemize}
\end{enumerate}

\textbf{Without attestation}: Attacker gains database access, exfiltrates customer data

\textbf{With attestation}: Attack detected immediately, zero data loss

\subsubsection{Benefits Demonstrated}

\begin{itemize}[leftmargin=*]
    \item \textbf{Boot-Level Security}: Detects firmware/bootloader malware
    \item \textbf{Compliance}: Cryptographic proof of approved configuration
    \item \textbf{Zero-Trust Architecture}: Never trust, always verify
    \item \textbf{Automated Response}: Real-time detection and denial
    \item \textbf{Cryptographic Assurance}: Cannot be bypassed by software
\end{itemize}

\subsection{Conclusion}

TPM-based remote attestation provides cryptographic proof of platform integrity, enabling trustworthy computing in distributed environments. By combining hardware roots of trust (EK), privacy-preserving credentials (AIK), and tamper-evident measurements (PCRs), attestation allows remote verifiers to make informed trust decisions. The enterprise cloud VM use case demonstrates how attestation prevents sophisticated boot-level attacks that evade traditional security tools.

\newpage
